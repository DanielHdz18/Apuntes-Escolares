\documentclass[10pt,letterpaper]{article}
\usepackage{polyglossia} %configuracion de idioma
\setmainlanguage{spanish} %elegir idioma español
\usepackage{fontspec} %paquete para cargar fuentes
\setsansfont{cmunb}[Extension=.otf,
UprightFont=*mr,
ItalicFont=*mo,
BoldFont=*sr, % semibold
BoldItalicFont=*so, % semibold oblique
NFSSFamily=cmbr]
\usepackage{cmbright} %fuentes
\setmainfont{Roboto}[Scale=1.0] %fuente Palatino Linotype
\setsansfont{Roboto}[Scale=1.0] %fuente Palatino Linotype
\usepackage{microtype} %mejorar generales al documento pdf
\usepackage{amsmath} %paquete matematicos
\usepackage{amsfonts} %fuentes matematicas varias
\usepackage{amssymb} %simbolos matematicos varios
\usepackage{graphicx} %incluir graficos
\usepackage[left=2cm, right=2cm, top=2cm, bottom=2cm]{geometry} %paquete de configuracion de margenes
\usepackage[dvipsnames]{xcolor} %paquete de colores
\usepackage[backend=biber, style=apa]{biblatex} %Biblo APA
\usepackage{epstopdf} %Agregar archivos svg
\usepackage{float}
\usepackage{multicol}
\title{El efecto de las condiciones de síntesis y la hidrofilicidad sintonizable en la capacidad de encapsulación de fármacos de las nanopartículas de PLA y PLGA}
\date{}
\author{Norbert Vargaa, Viktoria Hornoka, Laszlo Janovaka, Imre Dekanya, Edit Csapoa}
\begin{document}
	\maketitle
	
		\section{Introducción}
	\begin{multicols}{2}
	
	El desarrollo de sistemas  eficaces de administración de  fármacos  es  un  tema popular en la investigación farmacéutica  o  nanomedicina de hoy en  día  [1-3]. La    micro  o  nanoencapsulación  de  un  ingrediente farmacéutico  es   una  forma prometedora y ampliamente  utilizada de formulación  de  fármacos que permite  una  serie de interesantes conceptos novedosos   de  administración farmacéutica.  Por    ejemplo,en  aplicaciones de liberación controlada  de          fármacos, la  encapsulación mejora y prolonga la eficacia de los  ingredientes activos,   mientras que  en  la  orientación de   fármacos, las partículas de base polimérica se pueden  utilizar  como  portadores  para  la  administración  dirigida  del  fármaco a un sitio de acción específico [4],  Una  clase  de  la  amplia  gama  de  materiales  aplicados  son  los  polímeros   como  poliestireno  (PS), N-Isopropilacrilamida  (NIPAM) y PLA/PLGA [5,6],  La  excelente  biocompatibilidad,  bajo  costo  y   buenas propiedades  materiales del PLA  abriría   muchas  aplicaciones  en el campo médico,  como  los  sistemas de administración  de fármacos    [7-10]. Aquí se    informa de la técnica  de  preparación  y  caracterización detallada  de las propiedades del polímero  y  la masa molar.  Existen   varios  métodos  para preparar NP de PLGA,  incluido el  método de emulsión doble  simple  o sin él,  la precipitación salina,  etc. [11,12].   La nanoprecipitación sobresale debido  a  su  facilidad  de  procesamiento  y  reproducibilidad  y se aplica principalmente  para medicamentos hidrófobos    [4].  
	Se        probaron   tres tipos de medicamentos como moléculas modelo en portadores de PLGA  como  el ketoprofeno  (KP),  que  es un medicamento  antiinflamatorio  noesteroideo  (AINE)  utilizado principalmente en el  tratamiento  del dolor  agudo  y la artritis crónica  [13]. El  D-$\alpha$-tocoferol  más  hidrófobo  es  un  compuesto  orgánico derivado de la actividad  de la vitamina  E,  por lo que  se puede  utilizar  en la prevención  y el tratamiento  de algunas enfermedades  crónicas relacionadas con la edad  [14]. También   se aplicó un derivado soluble en  agua de  TP, D-$\alpha$-Tocoferol   polietilenglicol succinato 1000   (TPGS).  La vitamina  E  desarrolla  su  actividad como un  antioxidante que rompe la cadena  que  detiene las reacciones de los radicales  libres  [15]. Además, la  vitamina  E  encapsulada es  más  efectiva  contra  la oxidación   en comparación con  la vitamina E libre  [16]. Varias  publicaciones  se concentran  en  las  propiedades de liberación  de los NP PLGA  preparados  o  comprados, sin  embargo,  muy  pocos describen las propiedades   del  polímero  y  su  efecto  sobre  los  NP  formadores  y  su EE (\%) [17,18]. Nuestros  resultados    señalan  que  al  elegir  el  polímero apropiado, los copolímeros PAL  o  PLGA que tienen un 65\% (PLGA65) y un 75\% (PLGA75\%)  contenido de monómero de lactida,   el  EE (\%) puede mejorarse   significativamente,   además, a una concentración   dada,  el medicamento  TP  puede encerrarse    en la cubierta del polímero  proporcionando propiedades  favorables adicionales   en  la liberación.
	
	\section{Sección experimental}
	
	\subsection{Materiales}
	
	Lactida (3,6-Dimetil-1,4-dioxano-2,5-diona) y  glicólido  (1,4-dioxano-2,5-diona)  comprados  a  Sigma-Aldrich y  Estaño(II) 2-etil-hexanoato    (de  Alfa  Aesar) y 1-dodecanol (de Fluka) se utilizaron    en  la síntesis de  polímeros.   PLGA  con  diferentes  lactidas:    glicólido con diferentes proporciones (PLGA65, 65:35,  $M_w$ = 66.000-107.000 y PLGA75, 75:25,  $M_w$ = 40.000-75.000,  de  Sigma-Aldrich) y PLA ($M_w$ = 250.000,  de  Fluka) se utilizaron    como  referencia  en  este  estudio, alcohol polivinílico  (PVA) ($M_w$ = 72,000),  Pluronic F127, bromuro de hexadecilo-trimetilamonio    (CTAB),  ketoprofeno  (KP), D-$\alpha$-tocoferol  polietilenglicol  succinato 1000 (TPGS), y ( ± )-$\alpha$-tocoferol  (TP) que  se  obtuvo  de  Sigma-Aldrich. El agua  se  purificó  con un aparato de purificación  Millipore  (18,2 M$\Omega$cm).  
	
	
	\subsection{Síntesis de copolímeros  PLGA }
	
	Los copolimeros  PLGA/PLA  con  hidrofilicidad  variada  se sintetizaron    mediante  polimerización de apertura de anillo (ROP) [9,19,20]. La D,L-lactida  (PLA: 5.0015 g, PLGA75: 3.7520 g, PLGA65: 3.2513 g) y el glicólido  (PLA:  0 g, PLGA75: 1.2510 g, PLGA65: 1.7502 g) se colocaron    en un matraz de fondo  redondo con un catalizador  de octanoato de estaño al 0.02\%  (de la masa total de  los dímeros) y al 0.01\% del iniciador de 1-dodecanol.  El  estaño-octanoato cataliza  la  reacción  de policondensación  [10,21],  mientras que  el  1-dodecanol  funcionó  como  iniciador  y  tiene  doble papel, activa  el  catalizador, por otro  lado, ya  que  un  agente nucleófilo    participa  en el paso de apertura del  anillo. Después de   que se agregaron   los  componentes,   el  sistema  se  cerró y se colocó  al  vacío  para  eliminar  las  moléculas de agua.  Para  iniciar  la  reacción de policondensación,   el  matraz  se colocó en un  baño de aceite precalentado a 170-180 °C.  A favor  de  una mejor  homogeneización,  se utilizó  agitación   magnética  durante  la  síntesis. Después de  dos  horas,  los  productos  se  vertieron  en  un  matraz sellado en un  refrigerador a - 20 °C. El  vacío, la  alta  temperatura  y la agitación magnética  fueron factores cruciales  de  la  próspera   síntesis. Es    importante  tener  en cuenta  que  el  iniciador  y  el  catalizador  tienen  baja  toxicidad  [9,22],  Ambos  componentes se  utilizan  en  las industrias alimentaria,  farmacéutica  y cosmética.    Gracias  a  la  baja  toxicidad  y  concentración  la  presencia  del    estaño-octanoato y 1-dodecanol  no fueron  un  problema significativo.
		
	\subsection{Síntesis de  los PLGA/PNPs NPA}


	Los NP de EGLP/PLA se prepararon    por  método  de nanoprecipitación  [10,12,19,21,23-25]. En  el  primer  paso  se   disolvió   el PLGA/PLA sintetizado  (15 mg)  en  la  fase  orgánica (acetona  o  1,4-dioxano) (1,5  mL)y  los estabilizadores aplicados   (como CTAB,  Pluronic  F127  o  PVA) (1,5 mg) en  la  fase  acuosa  (15  mL).  
	Dependiendo de su  hidrofilicidad,  el  fármaco  (7,5 mg) se   añadió  a  la  fase  acuosa  (TPGS)  o  orgánica  
	(KP, TP). Luego  se  agregó  la  fase  orgánica gota a gota  (10  $\mu$L)  a  la  fase  acuosa  bajo  agitación  magnética  continua  (1000 rpm) a    temperatura ambiente. Las dispersiones  NP  obtenidas  se agitaron   (350 rpm)  durante  dos  días.   Debido  a  su  mayor  temperatura de ebullición,  el  1,4-dioxano  no se evaporó por  completo,  pero  en  el  curso de los pasos de  limpieza  adicionales    se  escapó.   Las  dispersiones preparadas se     añadieron   a 40  mL  de  agua MQ y  se centrifugaron a 12.000 rpm (t = 15 min, T = 25 °C). El  NP  fue  recuperado  por  redispersación en 40  mL  MQ de agua. Los  pasos de  lavado  se repitieron    dos  veces. La dispersión  de NP  purificada  fue  liofilizada
			
		
		
	\end{multicols}

\end{document}