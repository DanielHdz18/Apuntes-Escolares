\documentclass[11pt,letterpaper]{article}
\usepackage{polyglossia} %configuracion de idioma
\setmainlanguage{spanish} %elegir idioma español
\usepackage{fontspec} %paquete para cargar fuentes
\setsansfont{cmunb}[Extension=.otf,
UprightFont=*mr,
ItalicFont=*mo,
BoldFont=*sr, % semibold
BoldItalicFont=*so, % semibold oblique
NFSSFamily=cmbr]
\usepackage{cmbright} %fuentes
\setmainfont{Roboto}[Scale=1.0] %fuente Palatino Linotype
\setsansfont{Roboto}[Scale=1.0] %fuente Palatino Linotype
\usepackage{microtype} %mejorar generales al documento pdf
\usepackage{amsmath} %paquete matematicos
\usepackage{amsfonts} %fuentes matematicas varias
\usepackage{amssymb} %simbolos matematicos varios
\usepackage{graphicx} %incluir graficos
\usepackage[left=2cm, right=2cm, top=2cm, bottom=2cm]{geometry} %paquete de configuracion de margenes
\usepackage[dvipsnames]{xcolor} %paquete de colores
\usepackage[backend=biber, style=apa]{biblatex} %Biblo APA
\usepackage[none]{hyphenat}
\usepackage{float}
\begin{document}
	
	\sloppy 
	
	\section{Introducción}
	
	The development of effective drug-delivery systems is a popular topic in todays’ pharmaceutical or nanomedicine-motivated research [1-3].  The   micro-   or   nanoencapsulation   of   a   pharmaceutical   ingredients is a promising a widely used way of drug formulation technology enabling a number of interesting novel pharmaceutical delivery concepts. For instance, in controlled drug release applications, encapsulation enhances and prolongs the effectiveness of active ingredients, while in drug targeting, polymeric-based particles can be used as carriers for the targeted delivery of the drug to a specific site of action [4], One class of the wide range of applied materials are the polymers like polystyrene (PS), JV-Isopropylacrylamide (NIPAM) and PLA/PLGA [5,6], The excellent biocompatibility, low cost and good material properties of PLA would open many applications in the medical  field,  such  as  drug  delivery  systems  [7-10].  Here  we  report  the preparation technique and detailed characterization of the polymer properties and molar mass. There are several methods to prepare PLGA NPs  including simple or  w/o/w double emulsion  method, salting-out
	etc. [11,12]. Nanoprecipitation excels due to its ease of processing and
	reproducibility and mainly applied for hydrophobic drugs [4].
	
	Three types of drugs as model molecules were tested in PLGA  carriers like ketoprofen  (KP), which is a non-steroidal anti-inflammatory drug (NSAID) mainly used in treatment of acute pain and chronic arthritis [13]. The more hydrophobic a-Tocopherol is an organic compound of vitamin E activity thus can be used in the prevention and treatment of some chronic, age-related diseases [14]. A water-soluble derivative of TP, D-a-Tocopherol polyethylene glycol 1000 succinate (TPGS)  was  also  applied.  Vitamin  E  evolves  its  activity  as  a chain breaking antioxidant  that  stops  the  free radical  reactions  [15]. Moreover, encapsulated vitamin E is more effective against oxidation as compared to free vitamin E [16]. Several publications concentrate on the release properties of prepared or purchased PLGA NPs, however, very  few  describes  the  polymer  properties  and  their  effect  on  the
	forming NPs and their EE (\%) [17,18]. Our results point out that by choosing the appropriate polymer, PLA or PLGA co-polymers having 65\%  (PLGA65)  and 75\%  (PLGA75\%) lactide monomer content, the EE
	(\%) can be significantly enhanced, moreover, at given concentration  the TP drug can be enclosed in the polymer shell providing further favorable properties in release.
	
	
	\section{Experimental section}
	
	\subsection{Materials}
	
	Lactide (3,6-Dimethyl-l,4-dioxane-2,5-dione) and glycolide (1,4- dioxane-2,5-dione) purchased from Sigma-Aldrich and Tin(II) 2-ethyl- hexanoate (from Alfa Aesar) and 1-dodecanol (from Fluka) were used in
	the  polymer  synthesis. PLGA  having different lactide :  glycolide com-
	ponent   ratio  (PLGA65,  65:35,  Mw = 66.000-107.000  and  PLGA75,
	75:25,  Mw = 40.000-75.000,   from  Sigma-Aldrich)   and   PLA  (Mw = 250.000, from Fluka) were used as reference in this study, polyvinyl alcohol (PVA) (Mw = 72,000), Pluronic FI27, hexadecyl- trimethylammonium bromide (CTAB), ketoprofen (KP), D-a-tocopherol polyethylene glycol 1000 succinate (TPGS), and ( ± )-a-tocopherol (TP)  was  obtained  from  Sigma-Aldrich.  Water  was  purified  with  a
	Millipore purification apparatus (18.2 M£!cm ).
	
	\subsection{Synthesis of PLGA copolymers}
	
	The PLGA/PLA co-polymers with varied hydrophilicity were synthesized by ring-opening polymerization (ROP) [9,19,20]. The D,L-lac- tide (PLA: 5.0015 g, PLGA75: 3.7520 g, PLGA65: 3.2513 g) and glycolide (PLA: Og, PLGA75: 1.2510 g,  PLGA65: 1.7502 g)  were put  into a
	round-bottomed  flask  with  0.02\%  tin-octanoate  catalyst  (of  the total dimers mass) and 0.01\% 1-dodecanol initiator. The tin-octanoate cat- alyzed the polycondensation reaction [10,21], while the 1-dodecanol functioned as initiator and has double role, it activates the catalyst on the other hand as a nucleophile agent takes part in the ring-opening step.  After  the  components  were  added,  the system  was  closed and
	placed  under  vacuum  in  order  to  remove  the  water  molecules. To initiate the polycondensation reaction, the flask was placed in a preheated  oil  bath  at  170-180 °C.  In  favor  of  better  homogenization, magnetic stirring was used during the synthesis. After two hours the
	products was poured into a sealed flask in a refrigerator at - 20 °C. The Vacuum, high temperature and magnetic stirring were crucial factors of the prosperous synthesis.
	It is important to note, that the initiator and the catalyst have low
	toxicity [9,22], Both components are used in the food, pharmaceuticals and cosmetics industries. Thanks to the low toxicity and concentration the presence of the tin-octanoate and 1-dodecanol were not signified problem.
	
	\subsection{Synthesis of PLGA/ PLA NPs}
	
	PLGA/PLA NPs were prepared by nanoprecipitation method [10,12,19,21,23-25].   In   the   first   step   the   synthetized   PLGA/PLA (15 mg) was dissolved in the organic phase (acetone or 1,4-dioxane) (1.5 mL)  and  the applied stabilizers  (such as CTAB, Pluronic F127 or
	PVA) (1.5 mg) in the aqueous phase (15 mL). Depending on their hydrophilicity, the drug (7.5 mg) was added to the aqueous (TPGS) or the
	organic phase (KP, TP). Then the organic phase was added dropwise (10 pL) to the aqueous phase under continuous magnetically stirring (1000 rpm) at room temperature. The obtained NP dispersions were stirred (350 rpm) for two days. Due  to its higher boiling temperature  the 1,4-dioxane not evaporated completely, but in the course of additional cleaning steps it got away. The prepared dispersions were added to  40 mL   MQ  water   and  centrifugated  at  12,000 rpm   (t = 15  min,
	T  = 25 °C).  The  NPs  was  retrieved  by  redispersation  in  40 mL MQ
	water. The washing steps were repeated two times. The purified NP dispersion was lyophilized
	
	
	\subsection{Polymer characterization methods}
	
	The DSC curves were recorded by Mettler Toledo DSC822e. In the test,  the  solid  samples  was  heated  from  25 °C  to  425 °C,  at  rate of
	5 °C- min-1  in a sealed Aluminum sample holder with a hole at the top.
	Nitrogen was used as the carrier gas at a flow rate of 50 mL- min-1. As a reference, the monomers was measured too. Infrared spectra of the monomers  and  polymers  were  measured  by  Jasco  FT/IR-4700  in  the attenuated  total  reflectance  (A'I'R)  mode  at  room  temperature  from 3500 cm-1 to 700 cm-1. The resolution was 2 cm 1 with 256 scans for all the samples.
	
	The size and molecular weight of PLGA/PLA NPs were determined by Horiba SZ-100. In both cases, the test was performed at scattering angle  of  90 °C  and   25 °C   temperature.  The  experimental  error determined for the average particle size from the parallel measurement was below 2.5\%. For the measurements of the molecular weight, the PLGA/PLA polymers was dissolved in 1,4-dioxane. Molecular weight (Mur) and second virial coefficient (AJ was identified by Rayleigh- Gans-Debye modell (Eq. (1)). Toluene was used as the reference dispersion solvent to  determine the extra  Rayleigh  ratio  (R       The concentration dependence of refractive index of each sample were determined by Mettler Toledo RM50 reffactometer which were important
	to  count  the  optical  constant  ( K ). Value  of  the structure  factor  ( P(0))
	was one for each cases. The C was the concentration of the polymer solutions in mgmL-1. The prediction intervals of Mw were determined from the linear fitting.
	
	The used solvents (MQ water, 1,4-dioxane) were filtered by 0.2 pm
	syringe filters (Flydrophobic  PTFE  0.2 pm, for 1,4-dioxane and Nylon
	0.2 pm for MQ water, Merck Kft., Germany Millipore Millex-FG, Merck
	Kft., Germany).
	Precipitation titrations of PLGA co-polymers were carried out at 25 °C with initial solution concentrations of 5 mgmL       1,4-dioxane as good solvent and water as poor solvent or precipitant. The change  in
	absorbance caused by the precipitation induced formation of polymer NPs, which was followed by UV-an Ocean Optics type USB4000 spectrophotometer in duplicate. For best palpability the absorbance percentage (A/Amax x 100 at  X  = 650 nm) were plotted in the function of the volume percentage of the water. The ratio of the amount of light absorption   to  the  absorption   at   complete  precipitation,   i.e., complete
	precipitation corresponds to 100\% turbidity. Also, the volume per cent
	precipitant has been calculated as if the volumes of solvent and precipitant were strictly additive. The monomer composition adjusted hydrophilicity of the synthesized polymers was determined by contact angle measurements. A portion of polymers was melted on  microscope
	glass slides at 160-170 °C in order to get smooth polymer film surface.
	The value of the water contact angle could be perfectly calculated with FM40Mk2 Easy Drop on the thus prepared surfaces at room temperature.  The  water  drops  were  added  with  a  syringe  equipped  with a
	0.5 mm diameter steel  needle. The volume of  the drops was 10 pL in
	every measurements. For the sake of comparison the analysis was also
	performed with the purchased PLGA/PLA.
	
	
	\subsection{Particle characterization methods}
	
	The TEM images were obtained in a Jeol JEM-1400plus equipment (Japan) at 120 keV accelerating voltage. The SEM measurements were performed by a field emission scanning electron microscope (FESEM, Hitachi S-4700) applying a secondary electron detector and 5 kV ac- celeration voltage.
	For the determination of EE (\%) in w/w \% the hydrophobic drugloaded PLGA/PLA NPs were dissolved in 1,4-dioxane and the concentration of the released drug were determined from the previously measured  calibration  curves  from  500 nm  to  200 nm  using  UV-vis spectrophotometer (Figs. SI and S2). The obtained data was divided by the total mass of drugs used in formulation (Eq. (2)).
	mass of drug in nanoparticles
	
	For the TPGS containing PLGA/PLA NPs the EE (\%) was determined
	indirectly (Fig. S3). The original dispersions (after the volume was set
	back to 15 mL) were centrifuged at 12,000 rpm (t = 15 min) and the UV
	spectra of the supernatants were used to determine the mass of the drug in NPs (Eq. (3)).
	
	\section{Results and discussion}
	
	\subsection{Characterization of polymers}
	
	The result of IR measurements confirmed the polymer formation.
	The peaks characteristic of the ester binds in the area denoted in Fig.1A
	around 1750 cm-1 is for C=0 bond, while the two at lower wavelengths (signed with blue and red lines in the online version) are for the
	C—O ester bond. The latter is similar in case of the polymers, while
	different from the spectra of the dimers. The C=0 stretching region
	appears in the IR spectra as a broad asymmetric band mainly due to A
	and El active modes especially for glycolide. The IR bands at 2997 and
	2949 cm-1 were assigned to the CH stretching region —CH3(asym),
	CH3(syrT1 j . The FTIR spectrum of a polymer in the fingerprint region
	( v < 1500 cm-1) is used to identify and characterize the material, since
	the observed peaks can be assigned to different vibration modes of
	chemical groups by comparison with cataloged FTIR spectra [10]. The CH3 is responsible for the appearance of the band at 1450 cm 1. The
	CH deformation and asymmetric bands appear at 1382 cm-1. The C—O
	stretching modes of the ester group appear at 1211 cm-1 and the
	C—O—C asymmetric mode appears at 1130 cm-1. At 956 and
	921 cm-1, we can find the bands characteristic of the helical backbone
	vibrations with the CH3 rocking modes. At 871 and 756 cm-1, two
	bands appear that can be attributed to the amorphous and crystalline
	phases of PLA, respectively.
	
	
	DSC measurements were carried out to control either the polymerization was successful. The thermogravimetric curves of the polymers were compared to the initial dimers1 (Fig. IB). PLA is a semicrystalline or amorphous polymer with a glass transit	ion ( Tg) and
	melting temperature (Tm) of approximately 55 °C and 180 °C, respectively. In our case the melting and glass transition temperature values
	are in accordance with the data found in literature [23]. The decomposition of the polymers took place at 304 °C for the pure PLA and
	systematically increases with decreasing lactide content to 341 °C for
	PLGA65. This is maybe due to the higher intermolecular interactions
	among the hydrophilic parts of the macromolecular chains. Characteristic thermal data of the purchased PLGA/PLA were also determined for comparison. The results were similar to the synthesized
	PLGA/PLA (Table 1). The thermal properties of the PLA can be varied
	by different parameters, mainly by the molecular weight and composition of D and L stereoisomers.

	The molar mass-related properties are essential, because they influence the particle-related parameters such as the diameter and the
	polydispersity. Therefore the molar weights are represented in Table 1.	
	For the purchased polymers are given as are presented in the compounds, while it was determined by DLS for the synthesized polymers.
	Based on the DLS results, the highest molar weight value was obtained
	for the less hydrophobic PLGA65 polymer. This is probably due to the
	higher glycolide formation rate. Significantly lower values were obtained for the initial hydrophobic PLA and the partially hydrophilized
	PLGA75 polymer (Fig. 2). The change in the refractive index was similar for the PLGA75 and 65 samples. Therefore the difference of the
	refractive indexes not influenced the DLS measurements either the
	molecular weight or the determination of the NP size. The value of the
	second virial coefficient corresponds to hydrophilicity of polymers,
	especially in case of PLGA65 and PLGA75. For PLA this value could he
	determined with higher error.
	The further investigations were related to the verification of the
	successful modification of polymer hydrophilicity. During the contact
	angle measurements, both the purchased and prepared polymers, the
	water was dropped on the form of melted polymer film with smooth
	surface. The results and the photos taken from the water drops were
	presented in Fig. 3A. The contact angle decreased from 74.55° to 68.18
	with decreasing amount of lactide part from 100\% (PLA) to 65\%
	(PLGA65). It proved that the hydrophilicitiy systematically change with
	the lactide:glycolide ratio. In addition, the contact angle are within the
	margin of error for our synthesized samples and the so called original,
	purchased one, which is demonstrated in Fig. 3A.
	Further details can be obtained from the titration (e.g. solventchange mediated precipitation) of polymer solution proceeding from
	good solvent (organic medium) with distilled water as poor solvent.
	According to the theory of Schulz, this fractional precipitation when we
	add non- solvent to a dilute polymer solution is also suitable for the
	determination of polymer molecular weight if the necessary constants
	are available [26]. Unfortunately, these constants were not found in the
	literature for our PLGA co-polymers and solvents, thus only the obtained precipitation curves can be presented without the corresponding
	molecular weights. The titration curve of our synthesized polymers
	were compared with the purchased ones, and the highest slope suggests tighter mass distribution in both co-polymers. The originally transparent solution became turbulent and polymer precipitation was observed without dissolving with mixing (Fig. 3B). Based on the curves we
	can state that our PLGA75 sample has smaller average mass than the
	purchased (Mw = 66-107 kDa), while our PLGA65 has higher
	(Mw = 40-70 kDa). The results are in accordance with the molecular
	weight obtained from light scattering.
	The turbidity of the polymer solution ( Le. the formation of polymer
	NPs in the solvent mixture) at the final state of the titration relative to
	the initially clear transparent solution also could be seen by naked eyes
	as the photos in Fig. 3B represents. This mechanism is also suitable for
	the encapsulation of (hydrophobic) drug particles into polymer shell
	[27]
	
	\subsection{Characterization of polymeric NPs}
	
	After the determination of polymer quality, we examined how the
	structure and the particle size of the PLGA/PLA NPs depend on the
	solvents (1,4 dioxane, acetone) and stabilizing agents (CTAB, PVA,
	Pluronic F127). The Fig. 4. presents the characteristic forms of the
	PLGA/PLA NPs with the three different stabilizing agents applied. Lowmagnification image for the samples with several particles can be inspected in the first line, while some particles for the same samples are
	magnified in the second line. CTAB surfactant as stabilizer resulted in
	the highest particles and particle-formation from the polymer was not
	complete. The NPs are sometimes attached to each other as oblivious
	for the Pluronic F127-stabilized one, though the PVA-stabilized seem
	less monodisperse relative to the Pluronic F127-stabilized NPs.
	The average particle diameter was determined both from TEM
	images and by DLS measurements. The Table summarizes the average
	particle diameter of the prepared NPs by different methods and stabilizing agent. Hereby the effect of solvent, lactide to glycolide ratio and
	stabilizing agents on the forming particle average diameter was investigated. It also can be seen, that the increasing GA content caused a
	decrease in the measured particles size in all cases
	As for the polymer composition, the PLA-based NPs showed the
	highest diameter. Their hydrophobic nature resulted in precipitates
	with higher diameter. The particles were smaller in every cases when
	the acetone was used as organic solvent for the preparation of NPs.
	Based on this observation, the type of the organic solvent significantly
	influences the particle size. The DLS results also verify the higher particle size of NPs if stabilized with the CTAB. The smallest particles were
	obtained for the Pluronic FI27 stabilized systems, these were the basis
	of our further drug-carrier studies as the preferred size of NPs are in the
	70-200 nm range for clinical application [28] due to the size of openings, also called fenestration of the endothelial barrier [29].
	For drug-containing NPs, the lyophilization of the samples is essential for further analysis and also for storing. The way and temperature of freezing followed by lyophilization is crucial to ensure the
	unchanging properties like particle size and dispersible properties. If
	the sample was frozen in an ultra-freeze drier at — 60 °C, the temperature distribution was not uniform, the structure was destructed
	resulting in attached spheres, and the particles were not redispersible. If
	liquid nitrogen was used dropwise to refrigerate the NPs under mixing,
	due to the homogeneous freezing, the particles were less aggregated.
	The difference can be discovered in both SEM and TEM images (Fig.
	S4).
	
	\subsection{Characterization of the drug-containing carrier systems}
	
	In order to utilize the PLA/PLGA NPs in drug delivery, it is important to determine the effects of drugs with different hydrophilicity
	(TPGS, KP, TP) on the nanoparticles. First of all the diameters of the
	PLGA/PLA NPs were measured by DLS. The particle size distribution
	curves of a few sample with a photo presenting one of the samples can
	be seen in Fig. S5. The drug-free samples show the more monodisperse
	size distribution relative to the drug-containing ones. The size enhancement due to the presence of drug is the most expressed in case of
	TP. The average particle diameter and encapsulation properties of different PLA/PLGA-drug systems are summarized in Table 3.
	Similar particle sizes were obtained for the drug-free particles regardless of the polymer type. The encapsulation of the drug increased
	the particle size especially in the case of TP as the efficiency values
	imply. The hydrophobic drug can be enclosed most efficiently in the
	polymers in these types of PLGA NPs. Presumably the more hydrophobic nature of the polymer supports the trapping of TP inside the
	polymer matrix in the aqueous medium. The TEM images explain the
	more expressed particle size enhancement in case of TP. The images
	clearly shows the TP is encapsulated inside the polymer forming a coreshell like structure (Fig. 5). This entrapped form of TP is advantageous
	for several reasons like the release measurements, where we can controlled and thus the dose frequency can be reduced to the drugs. Furthermore, the bioactivity and stability of the active substance entrapped
	in the NPs is protected by encapsulation and also the bioavailability was
	shown to increase which is critical for lipophilic components such as
	vitamin E [30]. No characteristic difference can be observed if TPGS or
	KP is encapsulated, but for TP the difference is striking.
	Highest EE (\%) was achieved for the most hydrophobic, TP drug.
	Core-shell type of encapsulation can be only achieved in our samples if
	the hydrophobic nature of the drug and the polymer is similar thus their
	precipitation coincide. In this case the drug is entrapped by the inner,
	more hydrophobic parts of the polymer during the nanoprecipitation. If
	the drug (like in case of KP and TPGS) is more hydrophilic than the
	polymer the precipitation of the drug is followed by the contraction of
	the polymer. The difference in KP and TPGS hydrophilicity is perfectly
	presented in the EE (\%) values of PLA (Table 3), where almost double
	amount of KP is encapsulated relative to the TPGS, while there extent
	are similar for PLGA75 carrier. The EE (\%) data also proves that the
	polymer structure affects the EE (\%) as it can be increased from 6.4 to
	11.3 for TPGS. Other parameters like drug concentration, drug to
	polymer ratio also effects the EE (\%) while entrapment of the hydrophilic drugs can only be ensured if the drug is attached to the polymer
	chains, which has low probability in diluted polymer solution. Both KP
	and TPGS can be adsorbed on the surface of the nanoparticles as illustrated in the sketches in Fig. 5.
	The distinction between the two types of encapsulation is represented in the drawings illustrating each drug-containing NP. Some of
	the drug is presumably entrapped in the case of KP or TPGS, not enough
	to form a continuous drug-phase inside the polymer chain-shell. Drug
	molecules adsorbed on the surface of NPs is probable for nanoprecipitation.
	
	\section{Conclusion}
	
	Both hydrophilic and hydrophobic drugs were encapsulated in PLA
	and PLGA co-polymers of different compositions. Our polymers have
	favorable properties relative to the purchased ones, with slightly different average molar mass and narrower weight distribution. The surface wetting properties changed according to the composition as the
	contact angles decreased with decreasing lactide content of the copolymer. The size of polymeric NPs can be tuned with the nanoprecipitation method in the 180-380 nm range by choosing of the adequate
	solvent or the stabilization agent. The CTAB as stabilizing agent resulted the highest average particle size and Pluronic FI27 the smallest
	one in every case. The encapsulation efficiency for the most hydrophobic tocopherol is outstanding and was best for PLGA75 sample.
	High EE (\%) (around 90\%) was also observed in PLA or PLGA polymers
	in other studies, but not in the TP-PLGA NPs [18,20,30]. Unlike TPGS
	and KP, the TP can be evolved the core-shell structure with PLGA/PLA.
	The key factor to obtain entrapped drug in the polymer is the simultaneous precipitation of the components. The EE (\%) can be further
	enhanced by the appropriately chosen co-polymer composition and
	solution composition (drug concentration and drug to polymer ratio) as
	presented in this study.
	
	
\end{document}