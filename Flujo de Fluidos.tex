% !TeX TS-program = lualatex
\documentclass[11pt,letterpaper]{article}
\usepackage{polyglossia} %configuracion de idioma
\setmainlanguage{spanish} %elegir idioma español
\usepackage{fontspec} %paquete para cargar fuentes
\setsansfont{cmunb}[Extension=.otf,
UprightFont=*mr,
ItalicFont=*mo,
BoldFont=*sr, % semibold
BoldItalicFont=*so, % semibold oblique
NFSSFamily=cmbr]
\usepackage{cmbright} %fuentes
\usepackage{microtype} %mejorar generales al documento pdf
\usepackage{amsmath} %paquete matematicos
\usepackage{amsfonts} %fuentes matematicas varias
\usepackage{amssymb} %simbolos matematicos varios
\usepackage{graphicx} %incluir graficos
\usepackage[left=2cm, right=2cm, top=2cm, bottom=2cm]{geometry} %paquete de configuracion de margenes
\usepackage[dvipsnames]{xcolor} %paquete de colores
\usepackage[backend=biber, style=apa]{biblatex} %Biblo APA
\DeclareLanguageMapping{spanish}{spanish-apa} %Idioma Apa
\addbibresource{Biblioteca.bib} %Archivo .bib
\setlength{\bibitemsep}{0.6\baselineskip}
\begin{document}
	\section{Introducción}
	
	El método más común para transportar fluidos de un punto a otro es impulsarlo a través de un sistema de tuberías. Las tuberías de sección circular son las más frecuentes, ya que esta forma ofrece no solo mayor resistencia estructural sino también mayor sección transversal para el mismo perímetro exterior que cualquier otra forma. \parencite{crane}
	
	\section{Propiedades físicas de los fluidos}
	
	La solución de cualquier problema de flujo de fluidos requiere un conocimiento previo de las propiedades físicas del fluido en cuestión. A continuación se presentan algunas de ellas.
	
	\subsection{Viscosidad}
	
	La viscosidad expresa la facilidad que tiene un fluido para fluir cuando se le aplica una fuerza externa. La viscosidad absoluta de un fluido, es una medida de su resistencia al desplazamiento o  a sufrir deformaciones internas.
	
	\subsection{Densidad}
	
	La densidad de una sustancia es su masa por unidad de volumen.
	
	\printbibliography[title={Referencias}]
	
\end{document}